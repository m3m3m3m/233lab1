\hfill \newline
\textbf{Build a circuit} \newline
\phantom{ } All the equipment we need to build a circuit is wires, elements and an experimental board where we could plug wires and pins of elements in.\\
\phantom{ } Wires are used not only to build the circuit itself, but also to make a bridge between some node in a circuit and a probe. We need to count all the wires we need before we start to build a circuit, which includes wires of both usages. Then, we roughly estimate the length of each wire we need, so that we can cut the wires with accordingly length off quickly.\\
\phantom{ } Elements are the most important part.In this lab, we need to pick out resistors and capacitors with the correct value(R or C) out of our lab kit, then the left is simply plug the pins into proper holes on the board. There are 4 colorful lines on each resistor, which make a 4-Band-Code. To read its value, we only need to pay attention to the first three colors (the other is separated from these three). Each color refers to a number. If we let $a$ be the number of the first band, $b$ for the second and $c$ for the third, we will have a formula $\mathrm{R} = (10a+b)\times c\mathrm{\Omega}$. For a 10k$\mathrm{\Omega}$ resistor, we need brown-black-orange band. There are three numbers on each capacitor,let $a$ be the first two-digit number, and $b$ be the last number, we will have a formula $\mathrm{C} = a \times 10^b \mathrm{pF}$. For a 10nF capacitor, we need number 103.\\
\phantom{ } Board is like the basis of our circuit. There are a lot of regularly arranged holes on it. We can check the inboard connectivity by using the Cont button on the multimeter. If the multimeter makes a beep sound, then the two holes that probes are connected to are electrically connected.\\
\textbf{Use the oscilloscope} \newline
\phantom{ } We use an oscilloscope to draw the shape of the signals and do relative measurements. In this lab, we will need to detect 2 channels(input and output). So we can see two V-t plots with different colors on our oscilloscope. To save the plot we detect, we can use the SAVE/RECALL button and PRINT button. A USB driver is also needed to save the image.\\
\phantom{ } To do the measurements, we can use the MEASURE button to measure some simple value, such as amplitude, frequency, peak-to-peak etc. But in order to do more precise measurement, we will need to use the CURSOR button. We can choose the type of cursors, the exact cursor we are controlling, and the channel of measured signal with the nearest column of buttons to the screen.\\
\textbf{Use the function generator} \newline
\phantom{ } We use a function generator to generate a specified AC signals usually as the input of a system. We can choose the type of the signal such as Sine, Square, Ramp etc. with a column of buttons that are enclosed in a "Function" box. To set different parameters of the signal, we can use the nearest column of buttons to the screen from the top menu. After all details are set, we can connect the generator's probe and the circuit with wires and press On button on the generator. The background light will be on if the generator is working.\\